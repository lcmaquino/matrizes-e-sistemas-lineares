\documentclass[12pt,a4paper]{article}
\usepackage[utf8]{inputenc}
\usepackage[brazil]{babel}
\usepackage{graphicx}
\usepackage{amssymb, amsfonts, amsmath}
\usepackage{float}
\usepackage{enumerate}
\usepackage[top=1.5cm, bottom=1.5cm, left=1.25cm, right=1.25cm]{geometry}

\begin{document}
\pagestyle{empty}

\begin{center}
  \begin{tabular}{ccc}
    \begin{tabular}{c}
      \includegraphics[scale=0.25]{../../biblioteca/imagem/brasao-de-armas-brasil} \\
    \end{tabular} & 
    \begin{tabular}{c}
      Ministério da Educação \\
      Universidade Federal dos Vales do Jequitinhonha e Mucuri \\
      Faculdade de Ciências Sociais, Aplicadas e Exatas - FACSAE \\
      Departamento de Ciências Exatas - DCEX \\
      Disciplina: Matrizes e Sistemas Lineares. \quad Semestre: 2020/1\\
      Prof. Me. Luiz C. M. de Aquino\\
    \end{tabular} &
    \begin{tabular}{c}
      \includegraphics[scale=0.25]{../../biblioteca/imagem/logo-ufvjm} \\
    \end{tabular}
  \end{tabular}
\end{center}

\begin{center}
 \textbf{Avaliação II}
\end{center}

\textbf{Instruções}
\begin{itemize}
 \item Todas as justificativas necessárias na solução de cada questão devem estar presentes nesta avaliação;
 \item As respostas finais de cada questão devem estar escritas de caneta;
 \item Esta avaliação tem um total de 35,0 pontos.
\end{itemize}

\begin{enumerate}
  \item \textbf{[7,0 pontos]} Em cada item abaixo dê exemplo de uma matriz $4\times 4$ que atenda aos requisitos solicitados.
    \begin{enumerate}
      \item Todos os termos não são nulos e o determinante é igual a 5.
      \item Todos os termos são irracionais e o determinante é racional.
      \item Todos os termos não são inteiros e o determinante é inteiro.
      \item Todos os termos são negativos e o determinante é positivo.
    \end{enumerate}

  \item \textbf{[7,0 pontos]} Calcule o determinante da matriz $A$ de três formas distintas: pela
    definição geral; pelo método de Sarrus; por redução à matriz triangular
    superior. 
    $$A=\begin{bmatrix}
      5 & -10 & -1\\
      -2 & 5 & 2\\
      -3 & 6 & 1
    \end{bmatrix}$$

  \item \textbf{[7,0 pontos]} Seja a matriz
    $A=\begin{bmatrix} -13 & 36 \\ -\dfrac{9}{2} & 14 \end{bmatrix}$.
    Determine o valor de $\lambda$ tal que $\det(A-\lambda I) = 0$.

  \item \textbf{[7,0 pontos]} Sabe-se que o determinante da matriz
    $A=\begin{bmatrix}
      p & 0 & p\\
      2 & 5 & 4\\
      q & -q & q
    \end{bmatrix}$ é $\dfrac{1}{3}$ e que o determinante da matriz
    $B=\begin{bmatrix}
      1 & -1 & 1 \\
      5p + 5q & 0 & 5p + 5q \\
      -1 & 1 & 1
    \end{bmatrix}$
    é $\dfrac{49}{6}$. Determine o valor de $p$ e $q$.
    
  \item \textbf{[7,0 pontos]} Prove que se $A$ é uma matriz $n\times n$ e $\alpha$
    é um escalar, então $\det(\alpha A) = \alpha^n\det(A)$.
    (Sugestão: use o Princípio de Indução Finita).
  
  \end{enumerate}
\end{document}
