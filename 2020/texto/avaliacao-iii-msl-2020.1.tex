\documentclass[12pt,a4paper]{article}
\usepackage[utf8]{inputenc}
\usepackage[brazil]{babel}
\usepackage{graphicx}
\usepackage{amssymb, amsfonts, amsmath}
\usepackage{float}
\usepackage{enumerate}
\usepackage[top=1.5cm, bottom=1.5cm, left=1.25cm, right=1.25cm]{geometry}

\begin{document}
\pagestyle{empty}

\begin{center}
  \begin{tabular}{ccc}
    \begin{tabular}{c}
      \includegraphics[scale=0.25]{../../biblioteca/imagem/brasao-de-armas-brasil} \\
    \end{tabular} & 
    \begin{tabular}{c}
      Ministério da Educação \\
      Universidade Federal dos Vales do Jequitinhonha e Mucuri \\
      Faculdade de Ciências Sociais, Aplicadas e Exatas - FACSAE \\
      Departamento de Ciências Exatas - DCEX \\
      Disciplina: Matrizes e Sistemas Lineares. \quad Semestre: 2020/1\\
      Prof. Me. Luiz C. M. de Aquino\\
    \end{tabular} &
    \begin{tabular}{c}
      \includegraphics[scale=0.25]{../../biblioteca/imagem/logo-ufvjm} \\
    \end{tabular}
  \end{tabular}
\end{center}

\begin{center}
 \textbf{Avaliação III}
\end{center}

\textbf{Instruções}
\begin{itemize}
 \item Todas as justificativas necessárias na solução de cada questão devem estar presentes nesta avaliação;
 \item As respostas finais de cada questão devem estar escritas de caneta;
 \item Esta avaliação tem um total de 35,0 pontos.
\end{itemize}

\begin{enumerate}
  \item \textbf{[7,0 pontos]} Considere a matriz
    $A=
      \begin{bmatrix}
        1 & 0 & 8 \\
        1 & 2 & 3 \\
        2 & 5 & 3
      \end{bmatrix}$

    \begin{enumerate}
      \item Determine a inversa de $A$.
      \item Usando o item (a), resolva o sistema linear abaixo:
      $$\begin{cases}
        x +  8 z  =  4\\
        x  +  2y  +  3z  =  0\\
        2x  +  5y  +  3z  =  -2
      \end{cases}$$
    \end{enumerate}

  \item \textbf{[7,0 pontos]} Prove que se $A = P^{-1}DP$, então $A^n = P^{-1}D^nP$ para todo $n\in\mathbb{N}$.
    (Observação: por convenção considere que $M^0=I$, para toda matriz quadrada $M$.)
  
  \item \textbf{[7,0 pontos]} Vamos usar operações com matrizes para criptografar uma mensagem.
    Primeiro, converta cada letra da mensagem em um número, como indica
    a tabela abaixo. Cada grupo de três letras, formará uma linha da matriz
    de mensagem $M$, de ordem $3\times3$. Agora, escolha uma matriz
    inversível $S$, de ordem $3\times3$, para ser a chave da criptografia.
    Para determinar a mensagem criptografada $C$, calculamos $C=SM$.
    Já para recuperar a mensagem original, calculamos $M=S^{-1}C$.

    \begin{center}
      \begin{tabular}{|c|c|c|c|c|c|c|c|c|c|c|c|c|}
      \hline 
      A & B & C & D & E & F & G & H & I & J & K & L & M\\
      \hline 
      1 & 2 & 3 & 4 & 5 & 6 & 7 & 8 & 9 & 10 & 11 & 12 & 13\\
      \hline 
      \hline 
      N & O & P & Q & R & S & T & U & V & W & X & Y & Z\\
      \hline 
      14 & 15 & 16 & 17 & 18 & 19 & 20 & 21 & 22 & 23 & 24 & 25 & 26\\
      \hline 
      \end{tabular}
  \end{center}

  Considerando que a chave de criptografia é a matriz
    $S=\begin{bmatrix}
      1 & -2 & 2\\
      2 & -3 & 6\\
      1 & 1 & 7
    \end{bmatrix}$ e a mensagem criptografada é
    $C=\begin{bmatrix}
      -13 & -4 & 15 \\
      -4 & 3 & 80 \\
      48 & 28 & 147
      \end{bmatrix}$, qual é a mensagem original?
  
  \item \textbf{[7,0 pontos]} Prove que se $A_1$, $A_2$, \ldots, $A_{k-1}$, $A_k$ são matrizes
  invertíveis de mesma ordem, então
  $\left(A_1\cdot A_2 \cdot \ldots \cdot A_{k-1} \cdot A_k\right)^{-1} = A_k^{-1}\cdot A_{k-1}^{-1}\cdot\ldots\cdot A_2^{-1}\cdot A_1^{-1}$

  \item \textbf{[7,0 pontos]} Prove que se $D$ é uma matriz diagonal invertível, então
    $$\left[D^{-1}\right]_{ij} = \begin{cases}\dfrac{1}{[D]_{ij}},\,i = j\\ 0,\, i\neq j\end{cases}$$
    
  \end{enumerate}
\end{document}
