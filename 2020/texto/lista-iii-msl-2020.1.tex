\documentclass[12pt,a4paper]{article}
\usepackage[utf8]{inputenc}
\usepackage[brazil]{babel}
\usepackage{graphicx}
\usepackage{amssymb, amsfonts, amsmath}
\usepackage{float}
\usepackage{enumerate}
\usepackage[top=1.5cm, bottom=1.5cm, left=1.25cm, right=1.25cm]{geometry}

\DeclareMathOperator{\sen}{sen}

\begin{document}
\pagestyle{empty}

\begin{center}
  \begin{tabular}{ccc}
    \begin{tabular}{c}
      \includegraphics[scale=0.25]{../../biblioteca/imagem/brasao-de-armas-brasil} \\
    \end{tabular} & 
    \begin{tabular}{c}
      Ministério da Educação \\
      Universidade Federal dos Vales do Jequitinhonha e Mucuri \\
      Faculdade de Ciências Sociais, Aplicadas e Exatas - FACSAE \\
      Departamento de Ciências Exatas - DCEX \\
      Disciplina: Matrizes e Sistemas Lineares. \quad Semestre: 2020/1\\
      Prof. Me. Luiz C. M. de Aquino\\
    \end{tabular} &
    \begin{tabular}{c}
      \includegraphics[scale=0.25]{../../biblioteca/imagem/logo-ufvjm} \\
    \end{tabular}
  \end{tabular}
\end{center}

\begin{center}
  \textbf{Lista de Exercícios III}
\end{center}

\begin{enumerate}
  \item Em cada item abaixo dê exemplo de uma matriz $4\times 4$ que atenda aos requisitos solicitados.
    \begin{enumerate}
      \item Todos os termos não são nulos e o determinante é igual a 5.
      \item Todos os termos são irracionais e o determinante é racional.
    \end{enumerate}

  \item Dizemos que uma matriz $A$ é ortogonal quando sua transposta coincide
    com sua inversa (ou seja, quando $A^t= A^{-1}$). Considerando a matriz
    $R_\theta=
      \begin{bmatrix}
        \cos \theta & -\sen \theta  \\
        \sen \theta & \cos \theta
      \end{bmatrix}$,
    responda aos quesitos abaixo.

    \begin{enumerate}
      \item Determine $R_{\frac{\pi}{2}}$.
      \item Determine $R_{\left(-\frac{\pi}{2}\right)}$.
      \item Mostre que $R_\theta$ é ortogonal.
      \item Mostre que $R_\alpha R_{\beta}=R_{\alpha +\beta}$.
    \end{enumerate}

  \item Considere a matriz
    $A=
      \begin{bmatrix}
        1 & 0 & 8 \\
        1 & 2 & 3 \\
        2 & 5 & 3
      \end{bmatrix}$

    \begin{enumerate}
      \item Determine a inversa de $A$.
      \item Usando o item (a), resolva o sistema linear abaixo:
      $$\begin{cases}
        x +  8 z  =  4\\
        x  +  2y  +  3z  =  0\\
        2x  +  5y  +  3z  =  -2
      \end{cases}$$
    \end{enumerate}

  \item O termo $a_{ij}$ de uma matriz $A$ de ordem $10\times 10$ é tal que:
    $$a_{ij} = \begin{cases}i+j;\,i\geq j \\ 0;\,i<j\end{cases}.$$ O valor do
    determinante de $A$ é igual a:
    
    \begin{tabular}{lllll}
      (a) $2^{10}\left(10!\right)$. & (b) $2(10!)$. & (c) $10^2(10!)$. & (d) $2\left(10^{10}\right)$. & (e) $2^{55}$.
    \end{tabular}

  \item Calcule o determinante da matriz $A$ de três formas distintas: pela
    definição geral; pelo método de Sarrus; por redução à matriz triangular
    superior. 
    $$A=\begin{bmatrix}
      5 & -10 & -1\\
      -2 & 5 & 2\\
      -3 & 6 & 1
    \end{bmatrix}$$

  \item Sabe-se que o determinante da matriz
    $A=\begin{bmatrix}
      p & 2 & 2\\
      p & 4 & 4\\
      p & 4 & 1
    \end{bmatrix}$ é $-18$. Sendo assim, calcule
    o determinante da matriz
    $B=\begin{bmatrix}
      p & 2 & 2\\
      4 & p & 4\\
      4 & 1 & p
    \end{bmatrix}$.

  \item Seja a matriz
    $A=\begin{bmatrix} -13 & 36 \\ -\dfrac{9}{2} & 14 \end{bmatrix}$.
    Determine o valor de $\lambda$ tal que $\det(A-\lambda I) = 0$.

  \item Calcule a inversa das matrizes abaixo:

  $$A=\begin{bmatrix}
      6 & 2\\
      4 & 3
    \end{bmatrix},\,
    B=\begin{bmatrix}
      1 & -2 & 2\\
      2 & -3 & 6\\
      1 & 1 & 7
    \end{bmatrix}\textrm{ e }
    C=\begin{bmatrix}
      1 & 1 & 1 & 1\\
      1 & 2 & -1 & 2\\
      1 & -1 & 2 & 1\\
      1 & 3 & 3 & 2
    \end{bmatrix}$$

  \item  Prove que se $A = B^{-1}CB$, então $A^n = B^{-1}C^nB$ para todo $n\in\mathbb{N}$.
    (Observação: por convenção considere que $M^0=I$, para toda matriz quadrada $M$.)

  \item Vamos usar operações com matrizes para criptografar uma mensagem.
    Primeiro, converta cada letra da mensagem em um número, como indica
    a tabela abaixo. Cada grupo de três letras, formará uma linha da matriz
    de mensagem $M$, de ordem $3\times3$. Agora, escolha uma matriz
    inversível $S$, de ordem $3\times3$, para ser a chave da criptografia.
    Para determinar a mensagem criptografada $C$, calculamos $C=SM$.
    Já para recuperar a mensagem original, calculamos $M=S^{-1}C$.

    \begin{center}
      \begin{tabular}{|c|c|c|c|c|c|c|c|c|c|c|c|c|}
      \hline 
      A & B & C & D & E & F & G & H & I & J & K & L & M\\
      \hline 
      1 & 2 & 3 & 4 & 5 & 6 & 7 & 8 & 9 & 10 & 11 & 12 & 13\\
      \hline 
      \hline 
      N & O & P & Q & R & S & T & U & V & W & X & Y & Z\\
      \hline 
      14 & 15 & 16 & 17 & 18 & 19 & 20 & 21 & 22 & 23 & 24 & 25 & 26\\
      \hline 
      \end{tabular}
  \end{center}

  Considerando que a chave de criptografia é a matriz
    $S=\begin{bmatrix}
      1 & -2 & 2\\
      2 & -3 & 6\\
      1 & 1 & 7
    \end{bmatrix}$ e a mensagem criptografada é
    $C=\begin{bmatrix}
      -13 & -4 & 15 \\
      -4 & 3 & 80 \\
      48 & 28 & 147
      \end{bmatrix}$, qual é a mensagem original?
\end{enumerate}

\end{document}