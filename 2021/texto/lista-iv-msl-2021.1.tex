\documentclass[12pt,a4paper]{article}
\usepackage[utf8]{inputenc}
\usepackage[brazil]{babel}
\usepackage{graphicx}
\usepackage{amssymb, amsfonts, amsmath}
\usepackage{float}
\usepackage{enumerate}
\usepackage[top=1.5cm, bottom=1.5cm, left=1.25cm, right=1.25cm]{geometry}

\DeclareMathOperator{\sen}{sen}

\begin{document}
\pagestyle{empty}

\begin{center}
  \begin{tabular}{ccc}
    \begin{tabular}{c}
      \includegraphics[scale=0.25]{../../biblioteca/imagem/brasao-de-armas-brasil} \\
    \end{tabular} & 
    \begin{tabular}{c}
      Ministério da Educação \\
      Universidade Federal dos Vales do Jequitinhonha e Mucuri \\
      Faculdade de Ciências Sociais, Aplicadas e Exatas - FACSAE \\
      Departamento de Ciências Exatas - DCEX \\
      Disciplina: Matrizes e Sistemas Lineares. \quad Semestre: 2021/1\\
      Prof. Me. Luiz C. M. de Aquino\\
    \end{tabular} &
    \begin{tabular}{c}
      \includegraphics[scale=0.25]{../../biblioteca/imagem/logo-ufvjm} \\
    \end{tabular}
  \end{tabular}
\end{center}

\begin{center}
  \textbf{Lista IV}
\end{center}

\begin{enumerate}
  \item  Dizemos que uma matriz $A$ é ortogonal quando sua transposta coincide
    com sua inversa (ou seja, quando $A^t= A^{-1}$). Considerando a matriz
    $R_\theta=
      \begin{bmatrix}
        \cos \theta & -\sen \theta  \\
        \sen \theta & \cos \theta
      \end{bmatrix}$,
    responda aos quesitos abaixo.

    \begin{enumerate}
      \item Determine $R_{\frac{\pi}{2}}$.
      \item Determine $R_{\left(-\frac{\pi}{3}\right)}$.
      \item Mostre que $R_\theta$ é ortogonal.
      \item Mostre que $R_\alpha R_{\beta}=R_{\alpha +\beta}$.
    \end{enumerate}

  \item Determine o valor de $p$ para que o sistema abaixo seja SPD.
    $$\begin{cases}
      3x_1 - x_2 = 2 \\
      2x_1 + px_2 = -1
    \end{cases}$$

  \item Prove que se $A = B^{-1}CB$, então $A^n = B^{-1}C^nB$ para todo $n\in\mathbb{N}$.
    (Observação: por convenção considere que $M^0=I$, para toda matriz quadrada $M$.)

  \item Vamos usar operações com matrizes para criptografar uma mensagem.
    Primeiro, converta cada letra da mensagem em um número, como indica
    a tabela abaixo. Cada grupo de três letras, formará uma linha da matriz
    de mensagem $M$, de ordem $3\times3$. Agora, escolha uma matriz
    invertível $S$, de ordem $3\times3$, para ser a chave da criptografia.
    Para determinar a mensagem criptografada $C$, calculamos $C=SM$.
    Já para recuperar a mensagem original, calculamos $M=S^{-1}C$.

    \begin{center}
      \begin{tabular}{|c|c|c|c|c|c|c|c|c|c|c|c|c|}
      \hline 
      A & B & C & D & E & F & G & H & I & J & K & L & M\\
      \hline 
      1 & 2 & 3 & 4 & 5 & 6 & 7 & 8 & 9 & 10 & 11 & 12 & 13\\
      \hline 
      \hline 
      N & O & P & Q & R & S & T & U & V & W & X & Y & Z\\
      \hline 
      14 & 15 & 16 & 17 & 18 & 19 & 20 & 21 & 22 & 23 & 24 & 25 & 26\\
      \hline 
      \end{tabular}
  \end{center}

  Considerando que a chave de criptografia é a matriz
    $S=\begin{bmatrix}
      1 & -2 & 2\\
      2 & -3 & 6\\
      1 & 1 & 7
    \end{bmatrix}$ e a mensagem criptografada é
    $C=\begin{bmatrix}
      -13 & -4 & 15 \\
      -4 & 3 & 80 \\
      48 & 28 & 147
      \end{bmatrix}$, qual é a mensagem original?

\end{enumerate}
  
\begin{center}
  \textbf{Gabarito}
\end{center}

[1] (a)$\begin{bmatrix}
        0 & -1  \\
        1 & 0
      \end{bmatrix}$. 
(b) $\begin{bmatrix}
        \dfrac{1}{2} & \dfrac{\sqrt{3}}{2}  \\ \\
        -\dfrac{\sqrt{3}}{2} & \dfrac{1}{2}
      \end{bmatrix}$. 
(c) Sugestão: mostre que $R_\theta R_\theta^t = I$. 
(d) Sugestão: use as identidades para o seno e o cosseno da soma de arcos. 
[2] $p \neq -\dfrac{2}{3}$. 
[3] Sugestão: aplique o Princípio de Indução Finita. 
[4] $M =
  \begin{bmatrix}
    A & L & G \\
    L & I & N \\
    E & A & R
  \end{bmatrix}$ .

\end{document}