\documentclass[12pt,a4paper]{article}
\usepackage[utf8]{inputenc}
\usepackage[brazil]{babel}
\usepackage{graphicx}
\usepackage{amssymb, amsfonts, amsmath}
\usepackage{float}
\usepackage{enumerate}
\usepackage[top=1.5cm, bottom=1.5cm, left=1.25cm, right=1.25cm]{geometry}

\begin{document}
\pagestyle{empty}

\begin{center}
  \begin{tabular}{ccc}
    \begin{tabular}{c}
      \includegraphics[scale=0.25]{../../biblioteca/imagem/brasao-de-armas-brasil} \\
    \end{tabular} & 
    \begin{tabular}{c}
      Ministério da Educação \\
      Universidade Federal dos Vales do Jequitinhonha e Mucuri \\
      Faculdade de Ciências Sociais, Aplicadas e Exatas - FACSAE \\
      Departamento de Ciências Exatas - DCEX \\
      Disciplina: Matrizes e Sistemas Lineares. \quad Semestre: 2021/1\\
      Prof. Me. Luiz C. M. de Aquino\\
    \end{tabular} &
    \begin{tabular}{c}
      \includegraphics[scale=0.25]{../../biblioteca/imagem/logo-ufvjm} \\
    \end{tabular}
  \end{tabular}
\end{center}

\begin{center}
  \textbf{Lista I}
\end{center}

\begin{enumerate}
  \item Considere as seguintes matrizes:
    $$A = \begin{bmatrix}2 \\ -1 \\ 4\end{bmatrix},\,B = \begin{bmatrix}5 & 3 & -4\end{bmatrix}\textrm{ e }
    C = \begin{bmatrix}1 & -1 & 0 \\ 2 & 3 & 5 \\ -2 & 8 & 4\end{bmatrix}.
    $$

    Efetue as operações abaixo:
    \begin{enumerate}[(a)]
      \item $A + (BC)^t$.
      \item $AB + C$.
      \item $BCA$
      \item $A^tA + BB^t$.
    \end{enumerate}

  \item Na teoria de matrizes, $A^{n}$, $n\in\mathbb{N}^*,$ significa o produto
    $\overbrace{A\cdot A\cdot\ldots\cdot A}^{n\textrm{ vezes}}$. Considerando
    essa informação, verifique que a matriz 
    $A=\left[\begin{array}{rr}
      1 & \dfrac{1}{y}\\
      y & 1
    \end{array}\right]$, com $y\in\mathbb{R}^{*}$, é solução da equação $X^{2}=2X$.

  \item Seja a matriz 
    $S=\left[\begin{array}{rr}
      1 & 2\\
      0 & 1
    \end{array}\right]$. Determine a matriz $B$ tal que $B^{2}=S$.

  \item Dizemos que duas matrizes $A$ e $B$ comutam quando $AB=BA$. Sendo
    $B=\left[\begin{array}{rr}
      1 & 1\\
      0 & 1
    \end{array}\right]$, determine o formato de todas as matrizes $A$ que comutam com $B$.

  \item Dadas as matrizes:
    $A=
      \begin{bmatrix}
        1 & 3 \\
        2 & 5
      \end{bmatrix},
      \ \ \ 
    B=\begin{bmatrix}
      2 & x \\
      -1 & y
    \end{bmatrix}$ e 
    $C=\begin{bmatrix}
      9 & 3 \\
      3 & 1
    \end{bmatrix}$

    \begin{enumerate}
      \item Determine $x , y \in\mathbb{R}$ tais que $AC=BC$.
     \item Quando temos $AC=BC$, podemos ``cancelar'' $C$ e dizer que $A=B$? 
    \end{enumerate}

  \item  Sejam $A$ e $B$ duas matrizes quadradas de mesma ordem. Prove que:
    \begin{enumerate}
      \item $(A+B)^t = A^t + B^t$.
      \item $\left(B^t\right)^t = B$.
      \item $\left(kB\right)^t = k(B^t)$, com $k\in\mathbb{R}$.
      \item $A + A^t$ é simétrica.
      \item $A - A^t$ é antissimétrica.
    \end{enumerate}

  \end{enumerate}
  
\begin{center}
  \textbf{Gabarito}
\end{center}

[1] 
(a) $\begin{bmatrix}21 \\ -29 \\ 3\end{bmatrix}$ 
(b) $\begin{bmatrix}11 & 5 & -8 \\ -3 & 0 & 9 \\ 18 & 20 & -12\end{bmatrix}$ 
(c) $\begin{bmatrix} 62 \end{bmatrix}$ 
(d) $\begin{bmatrix} 71 \end{bmatrix}$ 
[2] Sugestão: calcule $A\cdot A$ e compare com $2A$. 
[3] $B = \begin{bmatrix}1 & 1 \\ 0 & 1\end{bmatrix}$ ou $B = \begin{bmatrix}-1 & -1 \\ 0 & -1\end{bmatrix}$ 
[4] $A = \begin{bmatrix}a & b \\ 0 & a\end{bmatrix}$, com $a, b\in\mathbb{R}$. 
[5]
(a) $x = 0$ e $y = 14$. 
(b) Não podemos, pois não necessariamente teremos $A = B$.
[6] Sugestão: observe os termos das matrizes.

\end{document}