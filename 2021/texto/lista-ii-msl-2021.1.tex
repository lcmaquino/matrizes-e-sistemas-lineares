\documentclass[12pt,a4paper]{article}
\usepackage[utf8]{inputenc}
\usepackage[brazil]{babel}
\usepackage{graphicx}
\usepackage{amssymb, amsfonts, amsmath}
\usepackage{float}
\usepackage{enumerate}
\usepackage[top=1.5cm, bottom=1.5cm, left=1.25cm, right=1.25cm]{geometry}

\DeclareMathOperator{\sen}{sen}

\begin{document}
\pagestyle{empty}

\begin{center}
  \begin{tabular}{ccc}
    \begin{tabular}{c}
      \includegraphics[scale=0.25]{../../biblioteca/imagem/brasao-de-armas-brasil} \\
    \end{tabular} & 
    \begin{tabular}{c}
      Ministério da Educação \\
      Universidade Federal dos Vales do Jequitinhonha e Mucuri \\
      Faculdade de Ciências Sociais, Aplicadas e Exatas - FACSAE \\
      Departamento de Ciências Exatas - DCEX \\
      Disciplina: Matrizes e Sistemas Lineares. \quad Semestre: 2021/1\\
      Prof. Me. Luiz C. M. de Aquino\\
    \end{tabular} &
    \begin{tabular}{c}
      \includegraphics[scale=0.25]{../../biblioteca/imagem/logo-ufvjm} \\
    \end{tabular}
  \end{tabular}
\end{center}

\begin{center}
  \textbf{Lista II}
\end{center}

\begin{enumerate}
  \item Em cada item abaixo dê exemplo de uma matriz $4\times 4$ que atenda aos requisitos solicitados.
    \begin{enumerate}
      \item Todos os termos não são nulos e o determinante é igual a 5.
      \item Todos os termos são irracionais e o determinante é racional.
    \end{enumerate}

  \item O termo $a_{ij}$ de uma matriz $A$ de ordem $10\times 10$ é tal que:
    $$a_{ij} = \begin{cases}i+j;\,i\geq j \\ 0;\,i<j\end{cases}.$$ O valor do
    determinante de $A$ é igual a:
    
    \begin{tabular}{lllll}
      (a) $2^{10}\left(10!\right)$. & (b) $2(10!)$. & (c) $10^2(10!)$. & (d) $2\left(10^{10}\right)$. & (e) $2^{55}$.
    \end{tabular}

  \item Calcule o determinante da matriz $A$ de três formas distintas: pela
    definição geral; pelo método de Sarrus; por redução à matriz triangular
    superior. 
    $$A=\begin{bmatrix}
      5 & -10 & -1\\
      -2 & 5 & 2\\
      -3 & 6 & 1
    \end{bmatrix}$$

  \item Sabe-se que o determinante da matriz
    $A=\begin{bmatrix}
      p & 2 & 2\\
      p & 4 & 4\\
      p & 4 & 1
    \end{bmatrix}$ é $-18$. Sendo assim, calcule
    o determinante da matriz
    $B=\begin{bmatrix}
      p & 2 & 2\\
      4 & p & 4\\
      4 & 1 & p
    \end{bmatrix}$.

  \item Seja a matriz
    $A=\begin{bmatrix} -13 & 36 \\ -\dfrac{9}{2} & 14 \end{bmatrix}$.
    Determine o valor de $\lambda$ tal que $\det(A-\lambda I) = 0$.

\end{enumerate}
  
\begin{center}
  \textbf{Gabarito}
\end{center}

[1] 
(a) $\begin{bmatrix}
		1 & 1 & 1 & 1 \\
		1 & 2 & 2 & 2 \\
		1 & 1 & 2 & 2 \\
		1 & 1 & 1 & 6 \\
	\end{bmatrix}$ 
(b) $\begin{bmatrix}
		\pi & \pi & \pi & \pi \\
		\pi & 1 + \pi & 1 + \pi & 1 + \pi \\
		\pi & \pi & 1 + \pi & 1 + \pi \\
		\pi & \pi & \pi & \dfrac{1}{\pi} + \pi \\
	\end{bmatrix}$. 
Obs.: esse exercício tem outras respostas válidas.  
[2] (a) $2^{10}(10!)$. Sugestão: note que a matriz é triangular superior. 
[3] $\det(A) = 2$. 
[4] $\det(B) = 7$. 
[5] $\lambda = 5$ ou $\lambda = -4$. 

\end{document}