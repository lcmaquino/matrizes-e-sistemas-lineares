\documentclass[12pt,a4paper]{article}
\usepackage[utf8]{inputenc}
\usepackage[brazil]{babel}
\usepackage{graphicx}
\usepackage{amssymb, amsfonts, amsmath}
\usepackage{float}
\usepackage{enumerate}
\usepackage[top=1.5cm, bottom=1.5cm, left=1.25cm, right=1.25cm]{geometry}

\DeclareMathOperator{\sen}{sen}

\begin{document}
\pagestyle{empty}

\begin{center}
  \begin{tabular}{ccc}
    \begin{tabular}{c}
      \includegraphics[scale=0.25]{../../biblioteca/imagem/brasao-de-armas-brasil} \\
    \end{tabular} & 
    \begin{tabular}{c}
      Ministério da Educação \\
      Universidade Federal dos Vales do Jequitinhonha e Mucuri \\
      Faculdade de Ciências Sociais, Aplicadas e Exatas - FACSAE \\
      Departamento de Ciências Exatas - DCEX \\
      Disciplina: Matrizes e Sistemas Lineares. \quad Semestre: 2021/1\\
      Prof. Me. Luiz C. M. de Aquino\\
    \end{tabular} &
    \begin{tabular}{c}
      \includegraphics[scale=0.25]{../../biblioteca/imagem/logo-ufvjm} \\
    \end{tabular}
  \end{tabular}
\end{center}

\begin{center}
 \textbf{Avaliação III}
\end{center}

\textbf{Instruções}
\begin{itemize}
 \item Todas as justificativas necessárias na solução de cada questão devem estar presentes nesta avaliação;
 \item As respostas finais de cada questão devem estar escritas de caneta;
 \item Esta avaliação tem um total de 35,0 pontos.
\end{itemize}

\begin{enumerate}
  \item \textbf{[5,0 pontos]} Classifique o seguinte sistema de equações lineares em SPD, SPI ou SI.
  $$
    \begin{cases}
      -x + y + z = 8 \\
      2x - y - 2z = 4 \\
      x - z = 10
    \end{cases}
  $$

  \item \textbf{[5,0 pontos]} Prove que se $A$ é uma matriz invertível, então
  $\det A^{-1} = \dfrac{1}{\det A}$.

  \item Dizemos que uma matriz $A$ é ortogonal quando sua transposta coincide
    com sua inversa (ou seja, quando $A^t= A^{-1}$). Considerando a matriz
    $R_\theta=
      \begin{bmatrix}
        \cos \theta & -\sen \theta  \\
        \sen \theta & \cos \theta
      \end{bmatrix}$,
    responda aos quesitos abaixo.

    \begin{enumerate}
      \item \textbf{[1,0 pontos]} Determine $R_{\frac{\pi}{4}}$.
      \item \textbf{[1,0 pontos]} Determine $R_{\left(-\frac{\pi}{3}\right)}$.
      \item \textbf{[3,0 pontos]} Mostre que $R_\theta$ é ortogonal.
      \item \textbf{[3,0 pontos]} Mostre que $R_\alpha R_{\beta}=R_{\alpha +\beta}$.
    \end{enumerate}

  \item \textbf{[7,0 pontos]} Seja o sistema de equações lineares:
  $$
  \begin{cases}
    mx + y = 2 \\
    x - y = p
  \end{cases}
  $$

  Determine o valor de $m$ e $p$ para os quais o sistema seja SI ou SPD.
    
  \item \textbf{[10,0 pontos]} Sejam as matrizes:
  $$
    D =
      \begin{bmatrix}
        2 & 0 & 0 \\
        0 & \sqrt{3} & 0 \\
        0 & 0 & -1 \\
      \end{bmatrix}\quad \textrm{ e }\quad
    P =
      \begin{bmatrix}
        1 & -2 & 2\\
        2 & -3 & 6\\
        1 & 1 & 7
      \end{bmatrix}.
  $$
  
  Se $A = P^{-1}DP$, então calcule a matriz $A^4$.

  \end{enumerate}
\end{document}
