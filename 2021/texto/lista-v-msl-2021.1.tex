\documentclass[12pt,a4paper]{article}
\usepackage[utf8]{inputenc}
\usepackage[brazil]{babel}
\usepackage{graphicx}
\usepackage{amssymb, amsfonts, amsmath}
\usepackage{float}
\usepackage{enumerate}
\usepackage[top=1.5cm, bottom=1.5cm, left=1.25cm, right=1.25cm]{geometry}

\DeclareMathOperator{\sen}{sen}

\begin{document}
\pagestyle{empty}

\begin{center}
  \begin{tabular}{ccc}
    \begin{tabular}{c}
      \includegraphics[scale=0.25]{../../biblioteca/imagem/brasao-de-armas-brasil} \\
    \end{tabular} & 
    \begin{tabular}{c}
      Ministério da Educação \\
      Universidade Federal dos Vales do Jequitinhonha e Mucuri \\
      Faculdade de Ciências Sociais, Aplicadas e Exatas - FACSAE \\
      Departamento de Ciências Exatas - DCEX \\
      Disciplina: Matrizes e Sistemas Lineares. \quad Semestre: 2021/1\\
      Prof. Me. Luiz C. M. de Aquino\\
    \end{tabular} &
    \begin{tabular}{c}
      \includegraphics[scale=0.25]{../../biblioteca/imagem/logo-ufvjm} \\
    \end{tabular}
  \end{tabular}
\end{center}

\begin{center}
  \textbf{Lista V}
\end{center}

\begin{enumerate}
  \item Seja o sistema de equações lineares:
  $$
  \begin{cases}
    mx + y = 2 \\
    x - y = p
  \end{cases}
  $$

  Determine o valor de $m$ e $p$ para os quais o sistema seja SI ou SPD.
  
  \item Determine o valor de $k$ para o qual o sistema abaixo seja SPD.
  $$
  \begin{cases}
    -y + kz = -2 \\
    x + y + z = 1 \\
    kx -2y + 4z = -5
  \end{cases}
  $$
 
 \item Sejam as matrizes
 $A = \begin{bmatrix} 
 1 & - 1 & 2 \\
 1 & 0 & 3 \\
 0 & 1 & -1 \\
 \end{bmatrix}$,  
 $B = \begin{bmatrix} 2 \\ -1 \\ 1\end{bmatrix}$ e 
 $C = \begin{bmatrix} 4 \\ 3 \\ 7\end{bmatrix}$.
 
 \begin{enumerate}
   \item Calcule $A^{-1}$.
   \item Use $A^{-1}$ para determinar a matriz $X$ que é solução da equação
   matricial $AX + B = C$.
 \end{enumerate}
 
 \item Sejam $A$ e $B$ matrizes invertíveis de mesma ordem. Prove que:
 \begin{enumerate}
   \item $\left(AB\right)^{-1} = B^{-1}A^{-1}$
   \item $\left(A^{-1}\right)^{-1} = A$
   \item $\left(A^{t}\right)^{-1} = \left(A^{-1}\right)^{t}$
   \item $\left(\alpha B\right)^{-1} = \dfrac{1}{\alpha}\left(B^{-1}\right)$, $\alpha \neq 0$
 \end{enumerate}
 
\end{enumerate}
  
\begin{center}
  \textbf{Gabarito}
\end{center}

[1] Para $m = -1$ e $p\neq -2$ é SI. Para $m\neq -1$ é SPD. 
[2] $\{k\in \mathbb{R}\,|\, k\neq -4\textrm{ e } k\neq 1\}$. 
[3] (a) $A^{-1} = \begin{bmatrix} 
 \dfrac{3}{2} & -\dfrac{1}{2} & \dfrac{3}{2} \\ \\
 -\dfrac{1}{2} & \dfrac{1}{2} & \dfrac{1}{2} \\ \\
 -\dfrac{1}{2} & \dfrac{1}{2} & -\dfrac{1}{2}
 \end{bmatrix}$. 
 (b) Usando $X = A^{-1}(C - B)$, obtenha 
 $X = \begin{bmatrix} 10 \\ 4 \\ -2\end{bmatrix}$. 
[4] Sugestão: dadas as matrizes $P$ e $Q$, verifique se $PQ = QP = I$, para 
concluir que $Q$ é a matriz inversa de $P$.

\end{document}