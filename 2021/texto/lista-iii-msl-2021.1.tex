\documentclass[12pt,a4paper]{article}
\usepackage[utf8]{inputenc}
\usepackage[brazil]{babel}
\usepackage{graphicx}
\usepackage{amssymb, amsfonts, amsmath}
\usepackage{float}
\usepackage{enumerate}
\usepackage[top=1.5cm, bottom=1.5cm, left=1.25cm, right=1.25cm]{geometry}

\DeclareMathOperator{\sen}{sen}

\begin{document}
\pagestyle{empty}

\begin{center}
  \begin{tabular}{ccc}
    \begin{tabular}{c}
      \includegraphics[scale=0.25]{../../biblioteca/imagem/brasao-de-armas-brasil} \\
    \end{tabular} & 
    \begin{tabular}{c}
      Ministério da Educação \\
      Universidade Federal dos Vales do Jequitinhonha e Mucuri \\
      Faculdade de Ciências Sociais, Aplicadas e Exatas - FACSAE \\
      Departamento de Ciências Exatas - DCEX \\
      Disciplina: Matrizes e Sistemas Lineares. \quad Semestre: 2021/1\\
      Prof. Me. Luiz C. M. de Aquino\\
    \end{tabular} &
    \begin{tabular}{c}
      \includegraphics[scale=0.25]{../../biblioteca/imagem/logo-ufvjm} \\
    \end{tabular}
  \end{tabular}
\end{center}

\begin{center}
  \textbf{Lista III}
\end{center}

\begin{enumerate}
  \item Resolva o sistema de equações lineares:
    $$\begin{cases}
      x +  8 z  =  4\\
      x  +  2y  +  3z  =  0\\
      2x  +  5y  +  3z  =  -2
    \end{cases}$$

  \item Um fabricante de móveis produz cadeiras, mesinhas de centro e mesas
    de jantar. Cada cadeira leva 10 minutos para ser lixada, 6 minutos
    para ser tingida e 12 minutos para ser envernizada. Cada mesinha de
    centro leva 12 minutos para ser lixada, 8 minutos para ser tingida
    e 12 minutos para ser envernizada. Cada mesa de jantar leva 15 minutos
    para ser lixada, 12 minutos para ser tingida e 18 minutos para ser
    envernizada. A bancada para lixar fica disponível 1.340 minutos por
    semana, a bancada para tingir 940 minutos por semana e a bancada para
    envernizar 1.560 minutos por semana. Quantos móveis devem ser fabricados
    (por semana) de cada tipo para que as bancadas sejam plenamente utilizadas?

  \item Considere as matrizes 
    $A = \begin{bmatrix} 1 & -1 & 2 \\ 4 & 1 & -1 \\ 1 & -1 & 1\end{bmatrix}$ e 
    $b = \begin{bmatrix} 4 \\ 1 \\ 2\end{bmatrix}$. Determine:
    
    \begin{enumerate}
      \item as matrizes $L = \begin{bmatrix} 1 & 0 & 0 \\ l_{21} & 1 & 0 \\ l_{31} & l_{32} & 1\end{bmatrix}$ 
        e $U = \begin{bmatrix} u_{11} & u_{12} & u_{13} \\ 0 & u_{22} & u_{23} \\ 0 & 0 & u_{33}\end{bmatrix}$ 
        tais que $A = LU$;
      \item a matriz $y_{3\times 1}$ tal que $Ly = b$;
      \item a matriz $x_{3\times 1}$ tal que $Ux = y$.
    \end{enumerate}
    
    Por fim, verifique que a matriz $x_{3\times 1}$ encontrada no item (c) 
    é solução da equação $Ax = b$.

  \item Sejam as matrizes $A_{n\times n}$, $x_{n\times 1}$ e 
    $\bar{0}_{n\times 1}$ (isto é, matriz nula de ordem $n\times 1$).
    Prove que se as matrizes $x_1$ e $x_2$ (ambas $n\times 1$) são
    soluções da equação $Ax = \bar{0}$, então a matriz $\alpha x_1 + \beta x_2$,
    com $\alpha,\,\beta\in\mathbb{R}$, também é uma solução dessa equação.

\end{enumerate}
  
\begin{center}
  \textbf{Gabarito}
\end{center}

[1] $x = 4$, $y = -2$ e $z = 0$. 
[2] $x = 50$, $y = 20$ e $z = 40$.  
[3]
(a) $L =
  \begin{bmatrix}
    1 & 0 & 0 \\
    4 & 1 & 0 \\
    1 & 0 & 1
  \end{bmatrix}$, 
$U =
  \begin{bmatrix}
    1 & -1 & 2 \\
    0 & 5 & -9 \\
    0 & 0 & -1
  \end{bmatrix}$. 
(b) $y =
  \begin{bmatrix}
    4 \\
    -15 \\
    -2
  \end{bmatrix}$. 
(c) $x =
  \begin{bmatrix}
    \dfrac{3}{5}\\ \\
    \dfrac{3}{5}\\ \\
    2
  \end{bmatrix}$. 

[4] Sugestão: desenvolva a expressão $A(\alpha x_1 + \beta x_2)$ para obter $\bar{0}$. 

\end{document}