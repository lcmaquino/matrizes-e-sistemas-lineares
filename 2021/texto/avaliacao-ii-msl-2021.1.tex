\documentclass[12pt,a4paper]{article}
\usepackage[utf8]{inputenc}
\usepackage[brazil]{babel}
\usepackage{graphicx}
\usepackage{amssymb, amsfonts, amsmath}
\usepackage{float}
\usepackage{enumerate}
\usepackage[top=1.5cm, bottom=1.5cm, left=1.25cm, right=1.25cm]{geometry}

\begin{document}
\pagestyle{empty}

\begin{center}
  \begin{tabular}{ccc}
    \begin{tabular}{c}
      \includegraphics[scale=0.25]{../../biblioteca/imagem/brasao-de-armas-brasil} \\
    \end{tabular} & 
    \begin{tabular}{c}
      Ministério da Educação \\
      Universidade Federal dos Vales do Jequitinhonha e Mucuri \\
      Faculdade de Ciências Sociais, Aplicadas e Exatas - FACSAE \\
      Departamento de Ciências Exatas - DCEX \\
      Disciplina: Matrizes e Sistemas Lineares. \quad Semestre: 2021/1\\
      Prof. Me. Luiz C. M. de Aquino\\
    \end{tabular} &
    \begin{tabular}{c}
      \includegraphics[scale=0.25]{../../biblioteca/imagem/logo-ufvjm} \\
    \end{tabular}
  \end{tabular}
\end{center}

\begin{center}
 \textbf{Avaliação II}
\end{center}

\textbf{Instruções}
\begin{itemize}
 \item Todas as justificativas necessárias na solução de cada questão devem estar presentes nesta avaliação;
 \item As respostas finais de cada questão devem estar escritas de caneta;
 \item Esta avaliação tem um total de 35,0 pontos.
\end{itemize}

\begin{enumerate}
  \item \textbf{[7,0 pontos]} Resolva o sistema de equações lineares:
    $$\begin{cases}
      8x +  z  =  4\\
      3x  +  2y  +  z  =  0\\
      3x  +  5y  +  2z  =  -2
    \end{cases}$$

  \item \textbf{[7,0 pontos]} Uma empresa de transporte possui três tipos de caixa:
  $A$, $B$ e $C$. Cada caixa pode transportar simultaneamente três tipos de produtos
  ($X$, $Y$ e $Z$) na quantidade descrita pela tabela abaixo. Com base nessas
  informações, quantas caixas de cada tipo são necessárias para transportar $590$ unidades
  de $X$, $255$ de $Y$ e $480$ de $Z$?
  
    \begin{table}[H]
      \centering
      \begin{tabular}{|c|c|c|c|}
        \cline{2-4} 
        \multicolumn{1}{c|}{} & $X$ & $Y$ & $Z$\\
        \hline 
        $A$ & 10 & 5 & 4\\
        \hline 
        $B$ & 6 & 3 & 8\\
        \hline 
        $C$ & 20 & 8 & 16\\
        \hline 
      \end{tabular}
    \end{table}

  \item \textbf{[7,0 pontos]} Um fabricante de móveis produz cadeiras,
    mesinhas de centro e mesas de jantar. Cada cadeira leva 9 minutos para
    ser lixada, 8 minutos para ser tingida e 10 minutos para ser envernizada.
    Cada mesinha de centro leva 12 minutos para ser lixada, 12 minutos para
    ser tingida e 15 minutos para ser envernizada. Cada mesa de jantar leva
    8 minutos para ser lixada, 7 minutos para ser tingida e 14 minutos para
    ser envernizada. A bancada para lixar fica disponível 2.550 minutos por
    semana, a bancada para tingir 2.365 minutos por semana e a bancada para
    envernizar 3.350 minutos por semana. Quantos móveis devem ser fabricados
    (por semana) de cada tipo para que as bancadas sejam plenamente utilizadas?
    
  \item \textbf{[7,0 pontos]} Determine a função $f$ polinomial do 2° grau tal
    que $f(0) + f(1) = -1$, $f(2) = f(-2)$ e $f(-1) = 0$.
    
  \item \textbf{[7,0 pontos]} Sejam as matrizes $A_{n\times n}$, $x_{n\times 1}$ e 
    $\bar{0}_{n\times 1}$ (isto é, matriz nula de ordem $n\times 1$).
    Prove que se as matrizes $x_1$ e $x_2$ (ambas $n\times 1$) são
    soluções da equação $Ax = \bar{0}$, então a matriz $\alpha x_1 + \beta x_2$,
    com $\alpha,\,\beta\in\mathbb{R}$, também é uma solução dessa equação.

  \end{enumerate}
\end{document}
