\documentclass[12pt,a4paper]{article}
\usepackage[utf8]{inputenc}
\usepackage[brazil]{babel}
\usepackage{graphicx}
\usepackage{amssymb, amsfonts, amsmath}
\usepackage{float}
\usepackage{enumerate}
\usepackage[top=1.5cm, bottom=1.5cm, left=1.25cm, right=1.25cm]{geometry}

\begin{document}
\pagestyle{empty}

\begin{center}
  \begin{tabular}{ccc}
    \begin{tabular}{c}
      \includegraphics[scale=0.25]{../../biblioteca/imagem/brasao-de-armas-brasil} \\
    \end{tabular} & 
    \begin{tabular}{c}
      Ministério da Educação \\
      Universidade Federal dos Vales do Jequitinhonha e Mucuri \\
      Faculdade de Ciências Sociais, Aplicadas e Exatas - FACSAE \\
      Departamento de Ciências Exatas - DCEX \\
      Disciplina: Matrizes e Sistemas Lineares. \quad Semestre: 2021/1\\
      Prof. Me. Luiz C. M. de Aquino\\
    \end{tabular} &
    \begin{tabular}{c}
      \includegraphics[scale=0.25]{../../biblioteca/imagem/logo-ufvjm} \\
    \end{tabular}
  \end{tabular}
\end{center}

\begin{center}
 \textbf{Avaliação I}
\end{center}

\textbf{Instruções}
\begin{itemize}
 \item Todas as justificativas necessárias na solução de cada questão devem estar presentes nesta avaliação;
 \item As respostas finais de cada questão devem estar escritas de caneta;
 \item Esta avaliação tem um total de 30,0 pontos.
\end{itemize}

\begin{enumerate}
  \item \textbf{[5,0 pontos]} Na teoria de matrizes, $A^{n}$, $n\in\mathbb{N}^*,$ significa o produto
    $\overbrace{A\cdot A\cdot\ldots\cdot A}^{n\textrm{ vezes}}$. Considerando
    essa informação, verifique que a matriz 
    $A=\begin{bmatrix}
      1 & \dfrac{1}{y}\\
      y & 1
    \end{bmatrix}$, com $y\in\mathbb{R}^{*}$, é solução da equação matricial $X^{2}=2X$.

  \item \textbf{[5,0 pontos]} Dadas as matrizes
    $A=
      \begin{bmatrix}
        1 & 1 \\
        5 & 18
      \end{bmatrix},
      \ \ \ 
    B=\begin{bmatrix}
      3 & 5 \\
      -2 & 4
    \end{bmatrix}$ e 
    $C=\begin{bmatrix}
      2 & x \\
      y & -6
    \end{bmatrix}$, determine:

    \begin{enumerate}
      \item $x , y \in\mathbb{R}$ tais que $AC=BC$.
     \item quando temos $AC=BC$, podemos ``cancelar'' $C$ e dizer que $A=B$? Explique sua resposta.
    \end{enumerate}

  \item \textbf{[5,0 pontos]} Sejam $A$ e $B$ duas matrizes quadradas de mesma ordem. Prove que:
    \begin{enumerate}
      \item $A + A^t$ é simétrica.
      \item $A - A^t$ é antissimétrica.
    \end{enumerate}

  \item \textbf{[4,0 pontos]} Em cada item abaixo dê exemplo de uma matriz $4\times 4$ que atenda aos requisitos solicitados.
    \begin{enumerate}
      \item Todos os termos não são nulos e o determinante é igual a $-6$.
      \item Todos os termos não são inteiros e o determinante é inteiro.
    \end{enumerate}
    
  \item \textbf{[4,0 pontos]} Sabe-se que o determinante da matriz
    $A=\begin{bmatrix}
      p & 2 & 2\\
      p & 4 & 4\\
      p & 4 & 1
    \end{bmatrix}$ é $-18$. Sendo assim, calcule
    o determinante da matriz
    $B=\begin{bmatrix}
      p & 2 & 2\\
      4 & p & 4\\
      4 & 1 & p
    \end{bmatrix}$.
    
  \item \textbf{[7,0 pontos]} Seja a matriz
    $A=\begin{bmatrix} 2 & -\dfrac{1}{3}\\ \\ -3 & 2\end{bmatrix}$.
    Determine o valor de $\lambda$ tal que $\det(A-\lambda I) = 0$.

  \end{enumerate}
\end{document}
